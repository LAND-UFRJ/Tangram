\documentclass[english]{article}

\usepackage{babel}

\usepackage{latexsym}
\usepackage{times}

% preamble for articles

\oddsidemargin= 0.0in
\textwidth=6.5in
\topmargin=0.0in
\textheight=609pt
\setlength{\unitlength}{0.5cm}

\begin{document}

\title{Automatic tests for TANGRAM-II}
\author{support$@$land.ufrj.br}
\date{}

\maketitle


\section{Introduction}


\subsection{Directory struture}
\label{sec:dir}

For performing the Tangram-II automatic tests, the following directory was built in

\begin{verbatim}
    $PREFIX/Tangram2/testbed
\end{verbatim}
    where: {\bf \$PREFIX} is the directory where Tangram-II is installed.

In this directory, one can find several sub-directories, each one with a model which are distributed with the main application. Therefore, the main objective is to test one or more features, see Subsection \ref{sec:models}.

Every sub-direcotory has two sub-directory nested, {\bf tangram2.0}, and {\bf tangram3.0}. These two folders are the referential results obtained with the respective Tangram-II version.

\subsection{Models and methods tested}

At this first automatic test release, the scripts will check for errors in chain generation procedure and stationary and transient solutions methods. This subsection covers the procedures and methods tested in each model.

\label{sec:models}

\begin{enumerate}
\item Availability Model: Stationary Method - GTH Block;
\item Bounded\_Reward Model: Transient Methods - Probabilities, Matrix Visualization, State Ordering Program;
\item Deterministic\_Server Model: Stationary Method - Non-Markovian;
\item Go\_Back\_N Model: Transient Method - Expected Cumulative Impulse Reward;
\item LiteralModel: Chain Geration with literal parameters;
\item MM1k Model: Transient Method - Expected Cumulative Rate Reward;
\item MM1k Model: Stationary Methods - Gauss,Jacobi, SOR, Power;
\item Operationaltime Model: Transient Method - Operational time and related measures;
\item Outputqueueing Model: Stationary Method - GTH;
\item RateRewardaprox: Transient Method - Cumulative Rate Reward method - Direct, Cumulative Rate Reward method - Iterative, Efficient Transient Probability Distribution method - Direct, Efficient Transient Probability Distribution method - Iterative;
\end{enumerate} 


\section{The test script}

The test script is named {\it test.sh} and are found in the model directory, e.g. {\it \$PREFIX/Tangram2/testbed/Availability/}. This script ``evaluates'' the model with the Tangram-II installed version and compares with the reference results, e.g. {\it \$PREFIX/Tangram2/testbed/Availability/tangram2.0/}.

Let's take a look at {\it \$PREFIX/Tangram2/testbed/Availability/test.sh} script

\begin{verbatim}
#!/bin/bash

BASENAME=availability
DIFF="diff --brief"

if [ $# = 3 ]; then
    
    if [ ! -d $2 ]; then
        mkdir -p $2
    fi    
    cd $2
    make -f $3 clean
    CONT=1 
    cp $1/$BASENAME.obj   .
    cp $1/$BASENAME.maxvalues .
    cp $1/$BASENAME.partition .
    echo -n " -->Solving with GTH method"
    make -f $3 gthblock-solv BASE=$BASENAME     
    $DIFF $1/$BASENAME.SS.gthb $BASENAME.SS.gthb
         if [ $? = 0 ];then
             echo "--> DIFF OK"
         else
             echo "--> DIFF FAILED"
             exit 1
         fi
         
    exit 0
else
    echo
    echo " Usage:"
    echo
    echo "    $0 <tangram_old_dir> <tangram_new_dir> <Makefile>"
    echo
    exit 1
fi
\end{verbatim}

in this case, the GTH solution method is tested. The script arguments specify the reference result diretory $<tangram\_old\_dir>$ , the directory were the evaluated results will stay $<tangram\_new\_dir>$, and the Makefile path $<Makefile>$ (see Appendix \ref{app:Makefile}). Note that these directory must be absolute, e.g. {\it \$PREFIX/Tangram2/testbed/Availability/}. Note that the script arguments are the same for all models test scripts.

\subsection{How the test script works}

First, the fundamental files to evaluate the test are copied to $<tangram\_new\_dir>$. These files are necessary to run the Tangram-II in shell script because they are generated by the Graphical User Interface
(see Section \ref{sec:interface}).

After this, the Makefile is executed. The {\it target} points out which solution method is evaluated, e.g. {\it gthblock-solv}. Finally, to compare the obtained results with the referential results, the diff program is used.

\section{Files generated by GUI}
\label{sec:interface}

This section list all necessary files to solution methods generated by Tangram-II graphical user interface. 

The key idea is to generate these files with the latest release (or user modified) Tangram-II GUI and compare with old ones. Each model needs some specific files. They are listed bellow

\begin{enumerate}
\item $<$MM1k$>$.maxvalues - Mathematical Model Generation (Chain Generation Icon)
\item $<$MM1k$>$.{init\_prob,intervals} - Analytical Model Solution
$\rightarrow$ Transient $\rightarrow$ 
Uniformization. Options: Initial State;Initial Time = 10 Final Time = 10 Num.
Points = 1.
\item $<$availability$>$.partition - Analytical Model Solution $\rightarrow$
Stationary $\rightarrow$ Exact (GTH block elimination) Options:  Initial State =
1 Block Size = 100  Quantity = 5 (Add); Initial State = 501 Block Size = 76
Quantity = 1.
\item $<$bounded\_reward$>$.reward\_levels. birth\_death.buffer: Analytical
Model Solution $\rightarrow$ Transient $\rightarrow$ Distributions $\rightarrow$
Bounded Cumulative Reward: Reward Levels: 10
\item
$<$server\_det.NM.$>${interest\_measures}: Analytical Model Solution
$\rightarrow$ Stationary $\rightarrow$ Non-Markovian Models: Options Choose
Variables: Server\_Queue\_Det.Queue (.interest\_measures). 
\item $<$raterewardaprox$>$.int\_rewd:Analytical Model Solution
$\rightarrow$  Transient $\rightarrow$ Expected Values $\rightarrow$ ESRA $\rightarrow$ Direct: Final Time = 1
Num. Points = 1
\item $<$raterewardaprox$>$.int\_rewi:Analytical Model Solution
$\rightarrow$  Transient $\rightarrow$ Expected Values $\rightarrow$ ESRA $\rightarrow$ Iterative: Initial
Time = 0 Final Time = 1 Num. Points = 1 Erlang Stages = 50
\item $<$raterewardaprox$>$.int\_direct:Analytical Model Solution
$\rightarrow$  Transient $\rightarrow$ ESPA $\rightarrow$ Direct:Final Time = 1
Num. Points = 1
\item $<$raterewardaprox$>$.int\_direct:Analytical Model Solution
$\rightarrow$  Transient $\rightarrow$ ESPA $\rightarrow$ Iterative:Initial
Time = 0 Final Time = 1 Num. Points = 1 Erlang Stages = 50
\end{enumerate}

Note that the non mentioned fields are irrelevant.
 
\section{Makefile}
\label{app:Makefile}

To ease the chain generation and solution evaluation a Makefile was developted (Makefile.Tangram2). Some of Makefile {\it targets} are

\begin{enumerate}
\item {\it chain}: This {\it target} is related with Markov chain generation step. The dependences are targets {\it getnames} and the files {\it (BASE).maxvalues (BASE).user\_code.c};
\item {\it gth-solv}: This {\it target} is used to test the stationary method GTH.
\end{enumerate}

For each on of solution methods, as well the Matrix Visualition , there is a {\it target} responsable for evaluating the Tangram-II solution.

\end{document}
